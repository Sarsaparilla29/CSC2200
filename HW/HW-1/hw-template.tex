
\documentclass[11pt]{article}

\usepackage{comment}


\pagestyle{empty}

\setlength{\oddsidemargin}{0in}
\addtolength{\oddsidemargin}{-0.0in}
\setlength{\textwidth}{0in}
\addtolength{\textwidth}{6.25in}
\setlength{\topmargin}{0in}
\addtolength{\topmargin}{-0.25in}
\setlength{\headheight}{0in}
\setlength{\headsep}{0in}
\setlength{\textheight}{0in}
\addtolength{\textheight}{9.5in}
\setlength{\footskip}{0in}
\setlength{\columnsep}{0.5in}



\date{}

\begin{document}

\thispagestyle{empty}


{\LARGE
\begin{center}
{\bf \textsf{CSC2200: Computer Science II \\
	Homework 1\\
	Winter 2021\\}}
\end{center}
}

\noindent {\Large {\bf \textsf{Student name:}} Jacob Bennett}
\vspace*{0.9cm}




\subsection*{{\bf \textsf{Problem 1}}}
Use mathematical induction to prove that $1\cdot 1! + 2\cdot 2! + \cdots + n\cdot n! = (n+1)!-1$
whenever $n$ is a positive integer.

\vspace*{0.2cm}
\noindent {\bf Solution:}

\noindent {\bf Base Case:}

\noindent $(1 + 1)! - 1 = (2)! - 1 = (2 \cdot 1) - 1 = 2 - 1 = 1$

\noindent $1 \cdot 1! = 1 \cdot 1 = 1$

\begin{comment}
\noindent $(2 + 1)! - 1 = (3)! - 1 = (3\cdot 2\cdot 1) - 1 = 6 - 1 = 5$

\noindent $2 \cdot 2! = 2\cdot (2\cdot 1) = 5$
\end{comment}

\noindent $1\cdot 1! + 2\cdot 2! + \cdots + n\cdot n! = (n+1)! - 1$

\noindent $n=k; 1\cdot 1! + 2\cdot 2! + \cdots + k\cdot k! = (k+1)! - 1$

\noindent $n = k + 1$

\noindent $1\cdot 1! + 2\cdot 2! + \cdots + k\cdot k! = (k + 1)! - 1$
\begin{comment}
\noindent $1\cdot 1! + 2\cdot 2! + \cdots + k\cdot k! + (k + 1)\cdot (k + 1)! = (k + 1)! - 1 + (k + 1)\cdot (k + 1)!$

\noindent $1\cdot 1! + 2\cdot 2! + \cdots + k\cdot k! + (k + 1)\cdot (k + 1)!=(k+1)! - 1 + (k+1)\cdot (k+1)!$


\noindent $1\cdot 1! + 2\cdot 2! + \cdots + k\cdot k! + (k + 1)\cdot (k + 1)!=(k+1)! - 1 + (k+2-1)\cdot (k+1)!$

\noindent $1\cdot 1! + 2\cdot 2! + \cdots + k\cdot k! + (k + 1)\cdot (k + 1)!=(k+1)! - 1 + (k+2)! - (k+1)!$

\noindent $1\cdot 1! + 2\cdot 2! + \cdots + k\cdot k! + (k + 1)\cdot (k + 1)!=(k+2)! - 1$
\end{comment}

\noindent $((k+1) + 1)! - 1 = (k + 2)! - 1$

\noindent $k=1; (1+2)! - 1 = (3)! - 1 = (3\cdot 2\cdot 1) - 1 = 6 - 1 = 5$

\noindent $=(k+1)!$




\subsection*{{\bf \textsf{Problem 2}}}
Use mathematical induction to prove that $\sum_{i=1}^{n}{\frac{1}{2^i}}=(2^n-1)/2^n$
whenever $n$ is a positive integer.

\vspace*{0.2cm}
\noindent {\bf Solution:}
write your solution here

\noindent $\sum_{i=1}^{1}{\frac{1}{2^1}}=(2^1-1)/2^1$

\noindent $0.5 = (1)/2$

\noindent $0.5 = 0.5$

\noindent $n=k; \sum_{i=1}^{k}{\frac{1}{2^k}}=(2^k - 1)/2^k$

\noindent $n=k+1; \sum_{i=1}^{k+1}{\frac{1}{2^{k+1}}}={\frac{2^{k+1} - 1}{2^{k+1}}}$




\subsection*{{\bf \textsf{Problem 3}}}
a) Simplify the expression:
$\log 3xy^2 -\log27 xy$

b) Solve for $x$:
$\log 4x + \log 3 = \log 12$\\
The logarithms in a) and b) are base 2.

\vspace*{0.2cm}
\noindent {\bf Solution:}
write your solution here






\subsection*{{\bf \textsf{Problem 4}}}
Write a recursive function that counts the number of times the integer 1 occurs in an
array of integers. Your function should have as arguments the array and the 
number of elements in the array.

\vspace*{0.2cm}
\noindent {\bf Solution:}
write your solution here


\noindent Example: how to write code in \LaTeX
\begin{verbatim}

int f(int array[], numElements)
{
	int count;
	if (numElements == 0){
		return count;
	}
	else if (x == 1) {
		count++;
		return f(array[], numElements - 1);
	}
	else
		return f(array[], numElements - 1);
	
}	

\end{verbatim}




\end{document}


	
